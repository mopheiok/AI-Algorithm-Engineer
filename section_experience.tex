% Awesome CV LaTeX Template
%
% This template has been downloaded from:
% https://github.com/huajh/huajh-awesome-latex-cv
%
% Author:
% Junhao Hua



%Section: Work Experience at the top
\sectionTitle{工作/项目经历}{\faCode}
 
\begin{experiences}
			
 \experience
    {现在}   {机器学习算法工程师}{ 中兴通讯-大数据二部}{ VMAX-R算法应用团队}
    {2017年10月} {
                    \vskip 0.05cm
                    \begin{itemize}\setlength{\itemsep}{0.15cm}
                        \item \textbf{无线精准规划}:利用手机与基站的测量报告数据完成基站选址推荐。采用DBScan算法对弱覆盖区域进行聚类,由自适应覆盖半径和规则进行该区域的站址推荐。其中采用KD-Tree对空间数据建立索引,优化整个加站流程。模型预测建设基站后覆盖率有2\textasciitilde4\%的提升效果;
                        \item \textbf{电信诈骗号码识别}:负责利用电信运营商的通信大数据研究和设计识别算法。通过爬取样本标签、提取用户的通信行为特征为识别算法建立决策树模型。模型精度达0.902,召回率达0.955,在局点验证中从给定的30个诈骗号码中识别出了25个号码;
                        \item \textbf{店铺选址推荐}:负责利用手机与基站的测量报告数据完成目标人群热力统计。其中由于手机与基站通信数据的特点,如何进行人车区分称为统计的难点。通过采用时间窗的方式完成人流热力统计,结合用户属性进行目标人群筛选。进而将目标人群密集区域推荐为店铺候选地址。
                    \end{itemize}
                  }
                    {DBScan, 决策树, 不平衡学习, Python, Spark, Kafka, Spark-Streaming}
  \emptySeparator
  \experience
    { 2017年9月} {算法、数据挖掘工程师}{ 中兴通讯-大数据二部}{ 数据挖掘团队}
    {2016年2月}    {
                    \vskip 0.05cm
                      \begin{itemize}\setlength{\itemsep}{0.15cm}
                        \item \textbf{网元异常检测}:采用无监督的pLSA/LDA模型推断后验概率P(动作|词)实现动作类别归类;
                        \item \textbf{景区人流热力分析}:以及采用监督学习(KNN,SVM)对每帧图像的字典分类;                    
                        \item \textbf{用户轨迹分析}:提出一种简单的投票(``voting'')方法 实现单相机中的多目标检测任务。
                        \item \textbf{手机基站PRB(物理资源块)利用率预测}:\faGithub: \link{https://github.com/huajh/action_recognition} {github.com/huajh/action\_recognition}                                                                                          
                      \end{itemize}
                    }
                    {动作识别, 聚类, GBDT, Voting, Bagging\&Boosting}
	
  \emptySeparator
  \experience
  {2016年1月} {数据采集、自动化部署架构核心成员}{中兴通讯-大数据二部}{VMAX项目质量团队}
  {2015年5月 }    {
                    \vskip 0.05cm
				  	\begin{itemize}\setlength{\itemsep}{0.15cm}
				  		\item \textbf{监控网站}:负责Python构建爬虫,采集项目持续构建数据(基于Jenkins生成数据),搭建监控网站;
				  		% 将GMM,student-t有限混合模型以及基于Dirichlet process的无限混合模型应用于聚类问题;
				  		\item \textbf{自动化部署架构}:负责大数据平台应用程序自动化部署架构开发
				  		% 用变分贝叶斯方法推断这三个模型,并推导出详细的算法流程;
				  		% \item 通过考虑laplacian graph 提升了以上三种方法的性能。
				  		% \item \faGithub: \link{https://github.com/huajh/variational_bayesian_clusterings} {github.com/huajh/variational\_bayesian\_clusterings}                                                                                    
				  	\end{itemize}
				  }
				  {Jenkins, urllib, Chef, Selenium, Python, Linux/Shell}

\end{experiences}
